Видно, что точки из центральной части, как и ожидается, лежат в зелёной зоне. Чем больше увеличивается ширина интервалов, тем больше точек из левой и правой части попадают в зелёную зону. При радиусе в $6 \cdot 10 ^ {-4}$ в обоих выборках точки из всех трёх частей попадают в зелёную зону.
Также видно, что строго внешние измерения встречаются только во второй выборке при радиусе интервала $10 ^ {-4}$, а выбросы были обнаружены при том же радиусе в первой выборке.

Как можно заметить, выбросы действительно оказались вне прогнозного коридора, что верно в соответствие с приведённой теорией.