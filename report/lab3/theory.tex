\subsection{Классификация измерений}

В задаче интервальной регрессии важно классифицировать измерения по влиянию на итоговому модель. Мы будем разделять измерения следующим образом.

\begin{itemize}
	\item Внутренние -- это такие измерения, добавление которых в существующую модель не изменяет её (её информационное множество)
	\item Внешние -- такие измерения, добавление которых в существующую модель изменяет её информационное множество
\end{itemize}

У внутренних и внешних измерений имеются важные частные случаи:

\begin{itemize}
	\item Граничные -- измерения, определяющие какой-либо фрагмент границы информационного множества задачи. Стоит заметить, что удаление из модели внутренних, но не граничных измерений, не изменит её
	\item Выбросы -- такие измерения, которые делают информационное множество пустым
\end{itemize}

Для того, чтобы определить, к какому классу принадлежит очередное измерение, достаточно соотнести его с прогнозом существующей модели в данной точке.

\begin{itemize}
	\item Внутреннее измерение полностью содержит в себе прогнозный интервал
	\item Граничное измерение имеет с ним общий конец
	\item Внешнее интервальное измерение не содержит в себе полностью прогнозный интервал
	\item Если пересечение внешнего интервального измерения с прогнозным интервалом пустое, то измерение -- это выброс
\end{itemize}

\subsection{Размах и относительный остаток}

Для дальнейшего анализа измерений введём следующие понятия.

\begin{definition}[\textbf{Размах (плечо)}]
	Размах -- величина, показывающая, как соотносится ширина прогнозного коридора и полученного интервала в данной точке:
	
	\begin{equation}
	\ell(x, \mathbf{y}) = \frac{\mathtt{rad} \Upsilon(x)}{\mathtt{rad} \mathbf{y}}
	\end{equation}
\end{definition}

\begin{definition}[\textbf{Относительный остаток}]
	Относительный остаток показывает, как соотносится расстояние между центром измерения и прогнозного коридора в данной точке и радиусом измерения:
	
	\begin{equation}
	r(x, \mathbf{y}) = \frac{\mathtt{mid} \mathbf{y} - \mathtt{mid} \Upsilon(x)}{\mathtt{rad} \mathbf{y}}
	\end{equation}
\end{definition}

Для внутренних измерений, содержащих в себе прогнозный интервал, выполняется неравенство:

\begin{equation}
|r(x, \mathbf{y})| \leq 1 - \ell(x, \mathbf{y})
\end{equation}

Точное равенство будет выполнено исключительно для граничных наблюдений.

Выбросы удовлеторяют условию:

\begin{equation}
|r(x, \mathbf{y})| > 1 + \ell(x, \mathbf{y})
\end{equation}

Интервальные измерения, у которых величина неопределённости меньше, чем ширина прогнозного интервала, то есть плечо больше единицы, оказывают сильное влияние на модель. Их называют \textbf{строго внешними}.

\subsection{Диаграмма статусов для интервальных измерений}

На диаграмме статусов в зелёной области лежат внутренние измерения, в жёлтой -- внешние, за вертикальной чертой $\ell = 1$ -- строго внешние измерения. Наблюдения, расположенные на границе зеленой зоны, являются граничными.


Диаграмма статусов строится по каждому каналу.
Для этого необходимо произвести следующие шаги:
\begin{enumerate}
	\item Выполняется кусочно-линейная интервальная регрессия
	\item Из обынтерваленных входных данных вычитается центральная часть полученной аппроксимации
	\item Строится прогноз на всю выборку по центральной регрессии. Его используем для вычисления плеча и относительного остатка
\end{enumerate}