Представленные способы позволяют успешно решать задачу интервальной регрессии. 

Модель линейной интервальной регрессии является более точной, нежели точечная, так как позволяет более корректно обынтервалить данные и найти такой коэффициент пропорциональности, при котором коэффициент Жаккара будет в 1.65 раз больше, чем в прошлом способе.

В то же время модель кусочно-интервальной регрессии позволяет получить ещё более качественную аппроксимацию: был получен коэффициент Жаккара, на 0.003 лучший, чем в прошлой модели.

При решении поставленной задачи были обнаружены баги в пакете glpk, с которыми удалось побороться путём изменения солвера.