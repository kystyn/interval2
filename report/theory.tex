\subsection{Простая линейная регрессия для вещественных данных}
Пусть заданы две последовательности $X = \{x_i\}_{i=1}^{n}, Y = \{y_i\}_{i=1}^{n}, \; x_i, y_i \in \mathbb{R} \; \forall i = \overline{1,n}$. \textbf{Простой линейной регрессией} для этих последовательностей называется функция:
\begin{equation}
f(x) = \beta_0 + \beta_1 \cdot x
\end{equation}
подобранная так, чтобы вектор $F = \{f(x_i)\}_{i=1}^{n}$ был в каком-то смысле максимально близок к вектору $Y$.

Таким образом, для решения задачи простой линейной регрессии необходимо найти коэффициенты $\beta_0, \beta_1$.
В зависимости от выбираемого метода поиска коэффициентов будет меняться и мера близости подобранной линейной функции к вектору $Y$.

В данной работе будет использоваться метод наименьших квадратов (МНК). Данный метод позволяет решить задачу простой линейной регрессии, поставив задачу минимизации
второй (евклидовой) нормы разности векторов F и Y:

\begin{equation}
\displaystyle \sum_{i=1}^{n} \|\beta_0 + \beta_1 x_i - y_i \|_2 \underset{\beta_0, \beta_1}{\longrightarrow} \min
\end{equation}

\subsection{Обынтерваливание данных для интервальной регрессии}
Поскольку показания датчиков обладают погрешностью, полученные данные на самом деле следует рассматривать как интервалы, центр которых совпадает со считанными показаниями, а радиус равен некоторой базовой погрешности $\varepsilon$, умноженной на вес $w_i$. $\varepsilon$ является константой.

Для каждого из наборов данных $X^{(1)}$ и $X^{(2)}$, прочитанных из соответствующих файлов, построим простую линейную регрессию на вещественных числах в результате чего получим аппроксимацию:

\begin{equation}
Lin_k(i) = a^{(k)}_i \cdot i + b^{(k)}_i, \; k \in \{1, 2\}, \; i=\overline{1,n}
\end{equation}

Определим для каждой из выборки вектор весов $W_k$ простым способом: если значение аппроксимирующей прямой $Lin_k$ в точке $i$ не попадает в интервал $x^{(k)}_i \pm \varepsilon$, то увеличим радиус интервала в $w^{(k)}_i$ раз так, чтобы $Lin_k(i)$ оказалось на одной из границ интервала.

После того, как мы получили два интервальных вектора из $\mathbb{IR}^n$, вычтем из $x_i^{(k)}$ ``наклонную'' составляющую $a_i^{(k)} \cdot i$, получив таким образом ``горизонтальные'' векторы, для которых будем находить искомый коэффициент пропорциональности $R_{21}$.

\subsection{Коэффициента Жаккара. Поиск $R_{21}$}
Коэффициент Жаккара позволяет оценить, насколько хорошо совмещаются друг с другом заданные интервалы $x_1, \dots, x_n$. Вычисляется путём деления длины интервала-пересечения на длину интервала объединения по формуле:
\begin{equation}
JK(x_1, \dots, x_n) = \frac{wid \left( \underset{i=\overline{1,n}}\bigcap x_i \right)}{wid \left(\underset{i=\overline{1,n}}\bigcup x_i\right)}
\end{equation}

Используя данный коэффициент, мы можем подобрать такой $R_{21} \in \mathbb{R}$, чтобы полученные интервалы $X_2$ и $R_{21} \cdot X_1$ были максимально совместны. Для этого необходимо вычислять коэффициент Жаккара для совокупности компонент этих векторов.

Таким образом, для того, чтобы найти $R_{21}$, необходимо задать нижнюю и верхнюю границы поиска $\underline{R}, \overline{R}$, а затем при помощи бинарного поиска найти точку максимума коэффициента Жакккара в зависимости от выбранного $R_{21}$.

Числа $\underline{R}, \overline{R}$ можно найти тривиально, поделив наименьшую верхнюю границу среди интервалов вектора $R_{21} \cdot X_1$ на наибольшую нижнюю границу среди интервалов вектора $X_2$ и, соответственно, наибольшую на наименбшую соответствующие границы.