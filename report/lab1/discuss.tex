Исходя из представленных графиков, можно судить о том, что все описанные в теории этапы выполнены правильно.
\begin{itemize}
\item Простая линейная регрессия и обынтерваливание проведены так, что каждый интервал содержит соответствующую точку аппроксимирующей прямой, при этом аппроксимирующая прямая лежит визуально близко к исходным данным.

\item В результате отсечения наклонной части действительно получились визуально горизонтальные графики.

\item График зависимости коэффициента Жаккара от искомого множителя ожидаемо имеет один локальный максимум. При этом видно, что оценка интервала $R_{21}$ с точки зрения меры Жаккара действительно очень грубая: значение коэффициента Жаккара в нижней оценке приблизительно равно -0.5. Данное число привело бы к абсолютно неприемлемому результату интервальной регрессии: хоть точечные наборы и получились бы визуально похожими, этого нельзя было бы сказать про интервалы. Учитывая характер полученных данных, важно удостовериться именно в максимальном совпадении интервалов.

\item На последнем рисунке видно, что значительная часть интервалов совпадает практически идеально, что также является показателем качественно выполненной работы.
\end{itemize}